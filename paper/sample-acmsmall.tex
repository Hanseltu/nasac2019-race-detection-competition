%%
%% This is file `sample-acmsmall.tex',
%% generated with the docstrip utility.
%%
%% The original source files were:
%%
%% samples.dtx  (with options: `acmsmall')
%% 
%% IMPORTANT NOTICE:
%% 
%% For the copyright see the source file.
%% 
%% Any modified versions of this file must be renamed
%% with new filenames distinct from sample-acmsmall.tex.
%% 
%% For distribution of the original source see the terms
%% for copying and modification in the file samples.dtx.
%% 
%% This generated file may be distributed as long as the
%% original source files, as listed above, are part of the
%% same distribution. (The sources need not necessarily be
%% in the same archive or directory.)
%%
%% The first command in your LaTeX source must be the \documentclass command.
\documentclass[acmsmall]{acmart}
\settopmatter{printacmref=false} % Removes citation information below abstract
\renewcommand\footnotetextcopyrightpermission[1]{} 
\pagestyle{plain} % removes running headers

% removes footnote with conference information in first column
%%
%% \BibTeX command to typeset BibTeX logo in the docs



\begin{document}


\title{Interrupt Data Race Detection via Variable Access Pattern Search based on LLVM
 }

\author{Haoxin Tu}
\email{trovato@corporation.com}
\author{Zhide Zhou}


\authornote{Both authors contributed equally to this research.}
\email{trovato@corporation.com}
%\orcid{1234-5678-9012}
%\author{G.K.M. Tobin}
%\authornotemark[1]
%\email{webmaster@marysville-ohio.com}
%\affiliation{%
%  \institution{Institute for Clarity in Documentation}
%  \streetaddress{P.O. Box 1212}
%  \city{Dublin}
%  \state{Ohio}
%  \postcode{43017-6221}
%}

%\renewcommand{\shortauthors}{Trovato and Tobin, et al.}

%%
%% The abstract is a short summary of the work to be presented in the
%% article.





%%
%% This command processes the author and affiliation and title
%% information and builds the first part of the formatted document.
\maketitle


\noindent{\textbf{ABSTRACT}}

%Interrupt-driven embedded software is widely used in safety critical systems such as aerospace, rail transit, and medical equipment. %In this paper, we designed a data race detection finite automaton model based on LLVM IR load/store instruction, which helps us to detect the defects in the single variable access sequence mode in the program. 
%Race conditions, occured due to interactions between application tasks and interrupt handlers, are triggered by the frequent using interrupts in these systems.
%To verify the validity of our model, we designed a tool called xx. The first step of this tool is to compile the source code into LLVM IR code. The second step is to model according to the load/store instruction in IR, and output the processing result after running.
%The experimental results show that our tool can effectively detect the four defects caused by the single variable access pattern in the code.
%Few works is focusing on the interrupt data race detection via variable access pattern search, not to mention the method based on LLVM. 
%In this paper, we design xxx, an automated framework that can detect race conditions in interrupt-driven embedded software.

Interrupt data race detection is critical for interrupt-driven software, 
since dangerous bugs may be caused by interrupt data race. In this paper, 
we propose a pattern search method to detect interrupt data race based on LLVM. 
First, we construct the behavor of the program related to the shared variable via 
the 'load' and 'store' instructions of LLVM. Then, a patter search algorithm is designed 
to detecte the bugy shared variable access pattern. We evaluate the proposed method on racebench, 
which reveals that the presented approach can precisely detect race conditions.

~\\[0.1em]


\noindent{\textbf{Keywords:}} Interrupt-driven Program, Data Race, LLVM

\section{Introduction}

this is a introduction

ACM's consolidated article template, introduced in 2017, provides a
consistent \LaTeX\ style for use across ACM publications, and
incorporates accessibility and metadata-extraction functionality
necessary for future Digital Library endeavors. Numerous ACM and
SIG-specific \LaTeX\ templates have been examined, and their unique
features incorporated into this single new template.

If you are new to publishing with ACM, this document is a valuable
guide to the process of preparing your work for publication. If you
have published with ACM before, this document provides insight and
instruction into more recent changes to the article template.

The ``\verb|acmart|'' document class can be used to prepare articles
for any ACM publication --- conference or journal, and for any stage
of publication, from review to final ``camera-ready'' copy, to the
author's own version, with {\itshape very} few changes to the source.


\section{Problem Definition}


\section{Model Construction}

%\section{Evaluation}

%\section{Conclusions}


\newpage
\bibliographystyle{ACM-Reference-Format}
\bibliography{test}








\end{document}
%\endinput
%%
%% End of file `sample-acmsmall.tex'.
